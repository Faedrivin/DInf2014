\documentclass{article}
\usepackage[T1]{fontenc}
\usepackage[utf8]{inputenc}
\usepackage{amsmath,amssymb}
\usepackage{tikz}
\usepackage{tikzsymbols}
\usepackage{tabularx}
\usepackage{lmodern}
\usepackage{xcolor}
\usepackage{algorithm2e}
\usepackage{listings}
\usepackage{titlesec}
\usepackage{hyperref}
\usepackage[ngerman]{babel}
\titlelabel{\thetitle.\ Aufgabe}
\usetikzlibrary{arrows,automata}
\setlength\parindent{0pt}

\begin{document}

\begin{center}
  \Large{Informatik D -- Klausur 2013 -- Joker \& Pinguin}

  \large{Sebastian Höffner, Andrea Suckro}
\end{center}

\section{}%1
\begin{itemize}
	\item Deterministischer Kellerautomat
  \item Linear beschränkter Automat
  \item Turingmaschine (Nicht- und Deterministisch)
  \item Endlicher Automat (Nicht- und Deterministisch)
\end{itemize}

\section{}%2
$a,b/c$
\begin{enumerate}
	\item[$a$:] Eingabe/zu lesendes Symbol ($a \in \Sigma$)
  \item[$b$:] Oberstes Symbol des Stacks ($b \in \Gamma$). Wird immer entfernt (häufig ist es deshalb in $c$ ganz rechts vorhanden, damit der Stack \emph{nicht verändert} wird).
  \item[$c$:] Symbolkette, die, rechts beginnend, auf den Stack gepusht werden.
\end{enumerate}

\section{}%3
Auffüllen nach rechts, beginnend mit\dots
\begin{itemize}
	\item \dots deterministisch kontextfrei
	\item \dots regulär $ab^+c$
	\item \dots regulär $(0|\dots|9)^*.(0|\dots|9)^*$
	\item \dots rekursiv aufzählbar
\end{itemize}

\section{}%4
$ba^*b(cc|dd)$

\section{}%5
\begin{align*}
S &\rightarrow abba | abAba \\
A &\rightarrow aAa | bAb | cAc | aa | bb | cc
\end{align*}

\section{}%6
\begin{tikzpicture}[->, auto, node distance=3cm]
	\node[initial,state]   (A) {};
  \node[state,accepting] (B) [right of=A] {};
  \node[state] (C) [right of = B] {};
  \path (A) edge [] node {$a/\square,R$} (B)
        (B) edge [bend left, above] node {$b/\square,R$} (C)
        (C) edge [bend left, below] node {$b/\square,R$} (B);
\end{tikzpicture}

\section{}%7
\begin{align*}
A_1 &\rightarrow aA_2 | aA_3 \\
A_2 &\rightarrow aA_4 | bA_2 | cA_3 | a\\
A_3 &\rightarrow dA_2 \\
A_4 &\rightarrow \epsilon
\end{align*}
Vereinfachen zu
\begin{align*}
S &\rightarrow aA | aB \\
A &\rightarrow a | bA | cB\\
B &\rightarrow dA \\
\end{align*}

\section{}%8
\subsection{}%a
Sei $L$ eine reguläre Sprache. Es gibt eine Zahl $n:=n(L)$, so dass sich \emph{jedes} Wort $z \in L$ mit $|z| \geq n$ so in drei Teile $u$, $v$, $w$ zerlegen läßt (d.h. z = uvw), dass die folgenden Eigenschaften gelten: $|v| \geq 1$, $|uv| \leq n$, $uv^iw \in L \text{ für alle } i \geq 0$.

\subsection{}%b
Wir nehmen an, $L = \{b^ia^jb^{i+j} | i,j\in \mathbb{N}\}$ sei regulär. Dann gelten die o.g. Eigenschaften.

Wir betrachten das Wort $z = b^na^nb^{2n}$. Dann ist $v = b^l$ und $u = b^{m-l}, 1 \leq l \leq m \leq n-1$. Wenn wir nun $v$ entfernen ($z' = uv^0w$), passt die Anzahl $b$s vor den $a$s nicht mehr zur Anzahl der $b$s danach, da $|u|+|a^n| = |b^{m-l}|+|a^n| = n+m-l < 2n = |b^2n|$. Es liegt also ein Widerspruch vor, somit ist $L$ nicht regulär.

\section{}%9
Im PCP dürfen Tupel mehrfach verwendet werden, d.h. es gibt nicht endlich viele Möglichkeiten/Permutationen, sondern unendlich viele.

\section{}%10
\begin{center}
\begin{tabularx}{\textwidth}{lllll|X}
P & NP & NP-v & NP-s & nichts & \\
\hline
$\square$ & $\boxtimes$ & $\boxtimes$ & $\boxtimes$ & $\square$ & SubsetSum \\
$\square$ & $\boxtimes$ & $\boxtimes$ & $\boxtimes$ & $\square$ & Wang Kachelung \\
$\boxtimes$ & $\square$ & $\square$ & $\square$ & $\square$ & Kreise in ger. Graph \\
$\square$ & $\square$ & $\square$ & $\square$ & $\boxtimes$ & Sortieren \\
$\boxtimes$ & $\square$ & $\square$ & $\square$ & $\square$ & Wortproblem DKF Sprachen \\
$\square$ & $\boxtimes$ & $\boxtimes$ & $\boxtimes$ & $\square$ & Spannbaum mit Gewicht $\leq k$ \\
$\square$ & $\square$ & $\square$ & $\square$ & $\boxtimes$ & kürzeste Rundtour \\
\end{tabularx}
\end{center}

\section{}%11
\subsection{}%a
Existiert eine Teilmenge $M'\subseteq M$ mit Gesamtwert $W \geq k_1$ und Gesamtgewicht $G \leq k_2$?

\subsection{}%b
\begin{enumerate}
	\item $\mathcal{X}$ muss NP-vollständig sein.
  \item $\mathcal{Y}$ muss in NP liegen.
\end{enumerate}
Die NP-Härte wird für $\mathcal{Y}$ gezeigt.

\section{}%12
\begin{itemize}
	\item nein
  \item nein
  \item ja (nein? Die, die wir kennen schon.)
  \item ja (nein weil "`pseudopolynomiell"'?)
  \item nein
  \item nein
\end{itemize}


\end{document}