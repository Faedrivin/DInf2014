\documentclass{article}
\usepackage[T1]{fontenc}
\usepackage[utf8]{inputenc}
\usepackage{lmodern}
\usepackage[ngerman]{babel}
\usepackage{amsmath, amssymb}
\usepackage{array}
\usepackage{graphicx}
\usepackage{tikz, tikzsymbols}
\usepackage{xcolor}
\usepackage{algorithm2e}

\usepackage{hhline}

\usetikzlibrary{arrows,automata,fit}
\setlength\parindent{0pt}

\begin{document}

\begin{center}
  \Large{Informatik \rotatebox[origin=c]{180}{D}\raisebox{0.05em}{:} Übungsblatt 10}

  \large{Sebastian Höffner, Andrea Suckro}
\end{center}



\section*{Aufgabe 10.1}
\subsection*{a)}
Der Algorithmus $\mathcal{A}$ berechnet $3^{3^x}$.


\subsection*{b)}
Das uniforme Kostenmaß berücksichtigt nicht, dass die Werte für $y$ sehr schnell sehr groß werden und die Multiplikation rasch über 32 oder 64 bit Zahlen hinausreicht.
Insbesondere ist dank $\log_3(\log_3(18.446.744.073.709.551.615)) \approx 3.366$ ab einer Eingabe von $x > 3.366$ ein Overflow gegeben.

Im uniformen Kostenmaß wäre die Laufzeit immer $\mathcal{O}(n)$, da aber die Funktion $3^{3^x}$ exponentiell wächst, müssen die Multiplikationen später länger dauern.


\subsection*{c)}




\section*{Aufgabe 10.2}

\subsection*{a)}
\begin{align*}
\left(\max\limits_{x\in\mathbb{R}^n}\left\{c^Tx\middle|Ax\leq b\right\}, k\right), z(\mathcal{L})\geq k
\end{align*}
Das heißt, das Entscheidungsproblem ist, ob $c^Tx \geq k$ mit $Ax \leq b$ erfüllt ist.

\subsection*{b)}
\begin{algorithm}
\end{algorithm}



\section*{Aufgabe 10.3}
\subsection*{a)}
Man kann die Erfüllbarkeit in linearer Zeit testen, indem man für alle einzelnen Konjunktionsterme prüft, ob sie vollständig \texttt{true} sind -- sobald das zutrifft, können wir abbrechen.

\subsection*{b)}



\end{document}

