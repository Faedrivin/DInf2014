\documentclass{article}
\usepackage[T1]{fontenc}
\usepackage[utf8]{inputenc}
\usepackage{lmodern}
\usepackage[ngerman]{babel}
\usepackage{amsmath, amssymb}
\usepackage{phonetic} % for reversed D
\usepackage{wasysym}  % for the notes
\usepackage{tikz, tikzsymbols}
\usepackage{xcolor}
\usetikzlibrary{arrows,automata,fit}
\setlength\parindent{0pt}
    
\begin{document}

\begin{center}
  \Large{Informatik \revD: Übungsblatt 4}

  \large{Sebastian Höffner, Andrea Suckro, Rasmus Diederichsen}
\end{center}



\section*{Aufgabe 4.1}
\begin{samepage}
Automat zur Addition zweier Binärzahlen.
\begin{center}
\begin{tikzpicture}[->, auto, node distance=3cm]
  \node[initial,state]   (Z1)               {$Z_1$};
  \node[state]           (Z2) [right of=Z1] {$Z_2$};
  \node[state]           (Z3) [right of=Z2] {$Z_3$};
  \node[state]           (Z4) [right of=Z3] {$Z_4$};
  \node[state]           (Z5) [below of=Z1] {$Z_5$};
  \node[state]           (Z6) [below of=Z2] {$Z_6$};
  \node[state,accepting] (Z7) [above of=Z2] {$Z_7$};

  \path (Z1) edge [bend left, pos=0.7] node {$0/\epsilon$} (Z5)
             edge [bend left, pos=0.7] node {$1/\epsilon$} (Z2)
             edge                      node {$\#/\epsilon$}(Z7)
        (Z2) edge [bend left, pos=0.3] node {$0/1$}        (Z1)
             edge                      node {$1/0$}        (Z3)
        (Z3) edge [bend left]          node {$0/\epsilon$} (Z6)
             edge [bend left]          node {$1/\epsilon$} (Z4)
             edge                      node {$\#/1$}       (Z7)
        (Z4) edge [bend left]          node {$0/0, 1/1$}   (Z3)
        (Z5) edge [bend left]          node {$0/0, 1/1$}   (Z1)
        (Z6) edge [pos=0.2]            node {$0/1$}        (Z1)
             edge [bend left, pos=0.2] node {$1/0$}        (Z3)
        ;
\end{tikzpicture}
\end{center}
\end{samepage}



\section*{Aufgabe 4.2}
Das Alphabet der Sprache ist $\Sigma = \left\{ \emptyset, (,\ ),\ |,\ { }^*,\ \sigma_1,\ ...,\ \sigma_k \right\}$.

Die kleinste Sprachklasse ist Typ 2, kontextfreie Sprachen, um die Klammerung (öffnende mit \textit{korrespondierender} schließender Klammer) korrekt abzubilden.

Eine Grammatik ist z.B.:
\begin{align*}
S \rightarrow \sigma_1\ |\ ...\ |\ \sigma_k\ |\ (SS)\ |\ (S|S)\ |\ (S)^*\ |\ \emptyset
\end{align*}



\section*{Aufgabe 4.3}
\subsection*{a)}
Wir nehmen an $L_1=\left\{\eighthnote^i\halfnote^{2i} | i \geq 0\right\}$ sei regulär, damit gelte für $L_1$ das Pumping Lemma. Sei $n$ die entsprechende Wortmindestgröße. Dann betrachten wir das Wort $z=\eighthnote^n\halfnote^{2n} \in L_1$. Es existiert nun eine Zerlegung $uvw=z$ für die gilt:

\begin{align}
|v| = l &\geq 1\\
|uv| &\leq n\\
\eighthnote^{n-l}\halfnote^{2n} &\notin L_1
\end{align}

$\eighthnote^{n-l}\halfnote^{2n}$ kann nicht in $L_1$ liegen, da $2(n-l) = 2n$ nur gilt wenn $l=0$ ist, was allerdings nach (1) nicht erfüllt ist. $L_1$ ist somit nicht regulär.
\setcounter{equation}{0}


\subsection*{b)}
Wir nehmen an $L_2=\left\{\eighthnote^i\halfnote^j | 0\leq i \leq j\right\}$ sei regulär, damit gelte für $L_2$ das Pumping Lemma. Sei $n$ die entsprechende Wortmindestgröße. Dann betrachten wir das Wort $z=\eighthnote^n\halfnote^j, j = n + x, x \geq 0 \in L_2$. Wobei $x$ den Unterschied in der Länge zwischen $i$ und $j$ repräsentiert. Es existiert nun eine Zerlegung $uvw=z$ für die gilt:

\begin{align}
|v| &\geq 1\\
|uv| &\leq n\\
\eighthnote^{n+x+1}\halfnote^{j} &\notin L_2
\end{align}

$\eighthnote^{n+x+1}\halfnote^{j}$ sollte nach dem Pumping Lemma in $L_2$ liegen, kann es allerdings nicht, da $n+x+1 > j $ ist. $L_2$ ist somit nicht regulär.
\setcounter{equation}{0}


\subsection*{c)}
Wir nehmen an $L_3=\left\{\quarternote^{k^2}| k\geq0 \right\}$ sei regulär, damit gelte für $L_3$ das Pumping Lemma. Sei $n$ die entsprechende Wortmindestgröße. Dann betrachten wir das Wort $z=\quarternote^{n^2} \in L_3$. Es existiert nun eine Zerlegung $uvw=z$ für die gilt:

\begin{align}
|v| &\geq 1\\
|uv| &\leq n\\
uv^*w &\in L_3 ?
\end{align}

Wenn $z=\quarternote^{n^2} \in L_3$ ist, dann ist das nächste Wort der Sprache $z=\quarternote^{(n+1)^2} \in L_3$. Der Abstand zwischen diesen zwei Wörtern beträgt $\quarternote^{2n+1}$. Da $v$ maximal $n$ lang sein kann, kann mit $v^*$ ein Wort erzeugt werden ($z'=\quarternote^{2n}$) was in diesen Abstand fällt und somit $\notin L_3$ ist. $L_3$ ist somit nicht regulär.



\section*{Aufgabe 4.4}
\subsection*{a)}
Sei $L$ eine reguläre Sprache. Es gibt eine Zahl $n := n(L)$, so dass alle Wörter $z \in L$ mit $|z| \geq n$ sich zerlegen lassen als $z = uvw$ mit den Eigenschaften:
\begin{itemize}
	\item $|v| \geq 1$
	\item $|v| \leq n$
	\item $uv*w \in L$
\end{itemize}

$L$ ist regulär, es existiert also ein endlicher Automat $A$ mit Zuständen \linebreak$Z = \left\{z_1, ..., z_i\right\}$, der $L$ beschreibt. Wähle $n := |Z|$. Bei der Abarbeitung von $z$ werden werden $|z|+1$ Zustände durchlaufen (inkl. Start).

Dabei wird nach Lesen des letzten Symbols von $u$ (gelesen: $z_1,...,z_k$) der selbe Zustand $z_k$ erreicht, wie nach Lesen des letzten Symbols von $uv$. Das heißt $|v|$ ist Anzahl der Zustände der Folge $z_k, ...,z_k$. Die Folge $z_k,...,z_i$ liest $w$.

Da $|v|$ nur ein Teil der Gesamtzustandsmenge ist, ist $|v| \leq n$ erfüllt.

\vspace{1em} \textit{Geht auch einfach unter Annahme des Pumping Lemmas: \\$|uv|\leq n \Rightarrow |u|+|v|\leq n \Rightarrow |v| \leq n$}


\subsection*{b)}
Das Pumping Lemma besagt $|uv| \leq n$. Also können wir folgern:
\setcounter{equation}{0}
\begin{align}
                 & |uv| \leq n \\
\Leftrightarrow\ & |u|+|v| \leq n \\
\Rightarrow    \ & |v| \leq n
\end{align}
Damit erfüllen alle Sprachen, die das Pumping Lemma erfüllen, auch das alternative Pumping Lemma.


\subsection*{c)}
Wir nehmen an $L$ sei regulär, damit gelten das Pumping Lemma sowie das alternative Pumping Lemma. Sei $n$ die entsprechende Wortmindestgröße. Dann betrachten wir jeweils das Wort $z$ mit je einer Zerlegung $uvw=z$ für die gilt:

\begin{itemize}
	\item $|v| \geq 1$
	\item $|uv| \leq n$ bzw. $|v| \leq n$
	\item $uv^*w \in L$
\end{itemize}

Für $k\ a$s generiert die Sprache auch $k\ b$s, d.h. die Wortlänge $|z|$ ist genau $|z| = n = 2k$.

\subsubsection*{Alternatives Pumping Lemma}
Wenn wir zeigen wollen, dass das Alternative Pumping Lemma gilt, müssen wir für alle Wörter der Sprache ($\forall w \in L$) zeigen, dass es mindestens eine Zerlegung gibt, die die Eigenschaften des Alternativen Pumping Lemmas erfüllt.

Dazu sei $w$ nun ein beliebiges Wort der Sprache mit $|w|\geq n$. Das bedeutet, dass es mindestens $\frac{n}{2}$ a's und $\frac{n}{2}$ b's in beliebiger Reihenfolge geben muss. Es gibt also insbesondere mindestens eine Stelle in dem Wort bei dem $a$ auf $b$ wechselt (oder andersherum). Wir wählen genau diese Stelle als $v$. Es ist nun klar, dass durch $v^*$ immer noch gleich viele a's und b's in dem Wort sind. Wir können diese Stelle immer in dem Wort wählen (egal wie weit hinten sie steht), da es keine Längenbegrenzung für $u$ gibt. Wir haben somit eine Zerlegung für $\forall w \in L$ gefunden, die die Eigenschaften des Alternativen Pumping Lemmas erfüllt.

\subsubsection*{Pumping Lemma}
Angenommen die Sprache ist regulär. Sei $z = a^nb^n$, mit $n$ der Wortmindestlänge. 
\setcounter{equation}{0}
\begin{align}
   u &= a^{n-l},\ l \geq 1 \\
   v &= a^l \\
|uv| &\leq n \\
   w &= b^n
\end{align}
Weil $uvw \in L$ muss nach $uv^*w \in L$ auch $uw = a^{n-l}b^n \in L$ sein. Da $uw$ jedoch kein Wort der Sprache ist, liegt ein Widerspruch vor. Die Sprache ist also nicht regulär.

\subsubsection*{Sonstige Fragen}
Die Sprache ist nicht regulär, da das Pumping Lemma einen Widerspruchsbeweis ermöglicht.

Das Pumping Lemma ist stärker als das alternative Pumping Lemma. Da beide Beweise ähnlich funktionieren ist das originale Pumping Lemma nützlicher.



\section*{Aufgabe 4.5}
\textit{Hinweis: Da wir in der Vorlesung gezeigt haben, dass $L_1 \cap L_2$ und $\overline{L_1}$ für die regulären Sprachen $L_i$ und $L_j$ ebenfalls regulär sind, ist auch $L_i \backslash L_j$ regulär, da der Differenzmengenoperator $\backslash$ auch als $L_1 \backslash L_2 = L_1 \cap \overline{L_2}$ geschrieben werden kann. Reguläre Sprachen sind also auch unter der Differenz geschlossen.}

\vspace{1em}

Sei $L = L_1 \cup L_2 = \left\{c^ja^ib^i | i,j \geq 0 \right\} \cup \left\{ a^jb^i |i,j \geq 0 \right\}$ regulär.

$L_2$ ist regulär, da für die Sprache mindestens der reguläre Ausdruck $a^*b^*$ existiert.
\begin{align*}
L_3 = L \backslash L_2 &= \left\{c^ja^ib^i | i,j \geq 0 \right\} \cup \left\{ a^jb^i |i,j \geq 0 \right\}\ \backslash\ \left\{ a^jb^i |i,j \geq 0 \right\}\ \\
                       &= \left\{c^ja^ib^i | i,j \geq 0 \right\} = L_1
\end{align*}

$L_1$ ist nach Annahme, dass $L$ regulär ist, weiterhin regulär. 

\vspace{1em}

Die Sprache $L_4$ über dem selben Alphabet können wir definieren als $L_4 = \left\{c^ia^jb^k | i \geq 1, j, k \geq 0 \right\}$. Sie ist ebenfalls regulär, da für sie mindestens der reguläre Ausdruck $c^+a^*b^*$ existiert.

Die Differenz $L_5 = L_1 \backslash L_4$ ist
\begin{align*}
L_5 = L_1 \backslash L_4 &= \left\{c^ja^ib^i | i,j \geq 0 \right\} \backslash \left\{c^ia^jb^k | i \geq 1, j, k \geq 0 \right\} \\
                         &= \left\{a^ib^i | i \geq 0 \right\}
\end{align*}

$L_5$ ist jedoch nach Beweis in der Vorlesung nicht regulär. Dies ist ein Widerspruch, aus dem folgt: Die Sprache $L$ ist nicht regulär.



\section*{Aufgabe 4.6}
Wenn die Sprache regulär ist, dann können wir z.B. einen endlichen Automaten oder regulären Ausdruck finden, der die Sprache $L = \left\{a^ib^j | i \geq j \geq 0, j \leq 2000 \right\}$ beschreibt.

\subsection*{Regulärer Ausdruck}
\begin{samepage}
In Übung 2 haben wir $\Smiley^x$ verwendet, um $x$ Wiederholungen von $\Smiley$ darzustellen. Mit der gleichen Schreibweise können wir die Sprache $L$ als regulären Ausdruck darstellen:

\begin{align*}
a^0a^*b^0|a^1a^*b^1|a^2a^*b^2|...|a^{2000}a^*b^{2000}
\end{align*}
\end{samepage}

\subsection*{NDEA}
Wir können die Sprache $L$ auch als NDEA darstellen. Angenommen die Bedingung für $j$ ist nicht $j \leq 2000$ sondern $j \leq 0$, $j\leq 1$ oder $j \leq 2$.

Dann sind gültige Automaten:

\begin{samepage}
$j \leq 0$:

\begin{center}
\begin{tikzpicture}[->, auto, node distance=2cm]
  \node[initial,state]   (Z1)               {};
  \node[state,accepting] (Z2) [right of=Z1] {};

  \path (Z1) edge              node {$\epsilon$} (Z2)
        (Z2) edge [loop right] node {a}          (Z2)
        ;
\end{tikzpicture}
\end{center}
\end{samepage}

\begin{samepage}
$j \leq 1$:

\begin{center}
\begin{tikzpicture}[->, auto, node distance=2cm]
  \node[initial,state]   (Z1)               {};
  \node[state]           (Z2) [right of=Z1] {};
  \node[state,accepting] (Z3) [below of=Z2] {};
  \node[state]           (Z4) [right of=Z2] {};
  \node[state,accepting] (Z5) [right of=Z4] {};

  \path (Z1) edge              node {$\epsilon$} (Z2)
             edge              node {$\epsilon$} (Z3)
        (Z2) edge              node {a}          (Z4)
        (Z3) edge [loop above] node {a}          (Z3)
        (Z4) edge [loop above] node {a}          (Z4)
             edge              node {b}          (Z5)
        ;
\end{tikzpicture}
\end{center}
\end{samepage}

\begin{samepage}
$j \leq 2$:

\begin{center}
\begin{tikzpicture}[->, auto, node distance=1.8cm]
  \node[initial,state]    (Z1)               {};
  \node[state]            (Z2) [right of=Z1] {};
  \node[state,accepting]  (Z3) [below of=Z2] {};
  \node[state]            (Z4) [right of=Z2] {};
  \node[state,accepting]  (Z5) [right of=Z4] {};
  \node[state]            (Z6) [above of=Z2] {};
  \node[state]            (Z7) [right of=Z6] {};
  \node[state]            (Z8) [right of=Z7] {};
  \node[state]            (Z9) [right of=Z8] {};
  \node[state,accepting] (Z10) [right of=Z9] {};

  \path (Z1) edge              node {$\epsilon$} (Z2)
             edge              node {$\epsilon$} (Z3)
             edge              node {$\epsilon$} (Z6)
        (Z2) edge              node {a}          (Z4)
        (Z3) edge [loop above] node {a}          (Z3)
        (Z4) edge [loop above] node {a}          (Z4)
             edge              node {b}          (Z5)
        (Z6) edge              node {a}          (Z7) 
        (Z7) edge              node {a}          (Z8)
        (Z8) edge [loop above] node {a}          (Z8)
             edge              node {b}          (Z9) 
        (Z9) edge              node {b}          (Z10)
        ;
\end{tikzpicture}
\end{center}

Diesen Automaten können wir nach dem selben Schema erweitern, solange wir $j \leq 2000$ erfüllen. \textit{(Ohne diese obere Schranke wäre der Automat nicht endlich und die Sprache nicht regulär.)}

\textbf{Die Sprache $L = \left\{a^ib^j | i \geq j \geq 0, j \leq 2000 \right\}$ ist also regulär.}
\end{samepage}



\end{document}