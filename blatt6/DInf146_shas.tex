\documentclass{article}
\usepackage[T1]{fontenc}
\usepackage[utf8]{inputenc}
\usepackage{lmodern}
\usepackage[ngerman]{babel}
\usepackage{amsmath, amssymb}
\usepackage{array}
\usepackage{phonetic} % for reversed D
\usepackage{wasysym}  % for the notes
\usepackage{tikz, tikzsymbols}
\usepackage{xcolor}
\usetikzlibrary{arrows,automata,fit}
\setlength\parindent{0pt}

\newcommand{\rpt}{%
        \raisebox{.2ex}{:}%
        \raisebox{-.4ex}{\rule{.1ex}{2.5ex}\,\rule{.2ex}{2.5ex}}}
\newcommand{\revrpt}{%
        \raisebox{-.2ex}{\rule{.2ex}{2.5ex}\,\rule{.1ex}{2.5ex}}%
        \raisebox{.2ex}{:}}
    
\begin{document}

\begin{center}
  \Large{Informatik \revD: Übungsblatt 5}

  \large{Sebastian Höffner, Andrea Suckro}
\end{center}



\section*{Aufgabe 6.1}
%\begin{center}
%\begin{tikzpicture}[->, auto, node distance=3cm]
  %\node[initial,state] (Z0)               {$Z_0$};
  %\node[state]         (Z1) [right of=Z1] {$Z_1$};
  %\node[state]         (Z2) [above right of=Z2] {$Z_2$};
  %\node[state]         (Z3) [below right of=Z2] {$Z_3$};
  %\node[state]         (Z4) [right of=Z3] {$Z_3$};
%
  %\path (Z1) edge              node {$a$} (Z2)
        %(Z2) edge [loop below] node {$b$}        (Z2)
             %edge              node {$c$} (Z3)
        %;
%\end{tikzpicture}
%\end{center}




\section*{Aufgabe 6.2}



\section*{Aufgabe 6.3}



\section*{Aufgabe 6.4}



\section*{Aufgabe 6.5}



\section*{Aufgabe 6.6}
Beim Algorithmus zur Umwandlung NDKA-AdLK in die KF-Grammatik können nur Regeln der folgenden Arten generiert werden:
\begin{enumerate}
	\item $V \rightarrow \epsilon$
  \item $V \rightarrow V$
  \item $V \rightarrow \Sigma \times V$
  \item $V \rightarrow \Sigma \times V \times V$
\end{enumerate}

Die Regeln 1, 3 und 4 sind automatisch in der GNF (1 nach Übungsblatt, 3 und 4 nach Definition).

Für Regeln der Form 2 muss eine Umformung durchgeführt werden, damit sie der Definition der GNF entspricht.

Diese Umformung ist eine Substitution nach folgendem Algorithmus:

Für eine Regel $V \rightarrow U$ mit den Regeln $U_i \rightarrow \left\{W\right\}\cup\left\{\Sigma\times W\right\}\cup\left\{\Sigma \times W \times W\right\}\cup\left\{\epsilon\right\}$ erstelle Regeln $V \rightarrow \left\{W\right\}\cup\left\{\Sigma\times W\right\}\cup\left\{\Sigma \times W \times W\right\}\cup\left\{\epsilon\right\}$ durch ersetzen der \textit{ersten} Variable mit der Folgeregel $U_i$. Wird die Folgeregel $U_i$ nicht weiter benötigt, entferne sie.
Wiederhole den Vorgang bis nur noch Regeln der Form $V \rightarrow \left\{\Sigma\times W\right\}\cup\left\{\Sigma \times W \times W\right\}\cup\left\{\epsilon\right\}$ vorhanden sind.

Beispiel:
\begin{align*}
\begin{array}{llll}
Schritt 1 & Schritt 2 & Schritt 3 & Schritt 4 \\
\hline
S \rightarrow sA     & S \rightarrow sA      & S \rightarrow sA & S \rightarrow sA\\
A \rightarrow aA | B & A \rightarrow aA | CD & A \rightarrow aA | DD | cD &A \rightarrow aA | dD | cD\\
B \rightarrow CD     & C \rightarrow D & C \rightarrow c &C \rightarrow c\\
C \rightarrow D      & C \rightarrow c & D \rightarrow d &D \rightarrow d\\
C \rightarrow c      & D \rightarrow d &&\\
D \rightarrow d      & & &\\
\end{array}
\end{align*}


\end{document}
