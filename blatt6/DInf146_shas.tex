\documentclass{article}
\usepackage[T1]{fontenc}
\usepackage[utf8]{inputenc}
\usepackage{lmodern}
\usepackage[ngerman]{babel}
\usepackage{amsmath, amssymb}
\usepackage{array}
\usepackage{phonetic} % for reversed D
\usepackage{tikz, tikzsymbols}
\usepackage{xcolor}   % e.g. colored boxes
\usepackage{listings} % source code
\usepackage{algorithm2e} % pseudo code
\usetikzlibrary{arrows,automata,fit}
\setlength\parindent{0pt}

\usepackage[normalem]{ulem}
\usepackage{cancel}
    
\begin{document}

\begin{center}
  \Large{Informatik \revD: Übungsblatt 6}

  \large{Sebastian Höffner, Andrea Suckro}
\end{center}



\section*{Aufgabe 6.1}
\begin{center}
\begin{tikzpicture}[->, auto, node distance=2cm]
  \node[initial,state] (Z0)               {$Z_0$};
  \node[state]         (Z1) [right of=Z0] {$Z_1$};
  \node[coordinate]                (C1) [above right of=Z1]{};
  
  \node[state]         (Z2) [above left of=C1] {$Z_2$};
  \node[state]         (Z3) [below right of=C1] {$Z_3$};
  \node[coordinate] (C2) [above right of=Z3]{};
  \node[state]         (Z4) [above of=C2] {$Z_4$};
  \node[state]         (Z5) [right of=Z4] {$Z_5$};
  \node[state]         (Z6) [below of=Z5] {$Z_6$};
  \node[state]         (Z7) [below of=Z4] {$Z_7$};
  \node[state]         (Z8) [below of=Z3] {$Z_8$};
  \node[state]         (Z9) [below of=Z1] {$Z_9$};
  \node[state]         (Z10) [above of=Z2] {$Z_{10}$};

  \path (Z0) edge                      node {$\epsilon,\$/\#\$$}                   (Z1)
        (Z1) edge                      node {$a,\star\in\Gamma/b\star$}            (Z2)
             edge [bend left,pos=0.45] node {$c,a/ba$}                             (Z3)
        (Z2) edge [bend left,sloped]   node {$a,b/\epsilon$}                       (Z3)
             edge [pos=0.3, sloped]    node {$c,b/cc$}                             (Z3)
             edge                      node{$\epsilon,\star\in\Gamma/\epsilon$}    (Z10)
        (Z3) edge [bend left]          node {$a,\star\in\Gamma/\epsilon$}          (Z1)
             edge [sloped,pos=0.8]     node {$c,a/cc$}                             (Z4)
             edge                      node {$\circ\in\Sigma,\circ/\circ\circ$}    (Z8)
        (Z4) edge                      node {$\epsilon,c/cc$}                      (Z5)
        (Z5) edge                      node {$\epsilon,c/ac$}                      (Z6)
        (Z6) edge                      node {$\epsilon,a/ba$}                      (Z7)
        (Z7) edge                      node {$\epsilon,b/ab$}                      (Z3)
        (Z8) edge                      node {$\epsilon,\circ\in\Gamma/\circ\circ$} (Z9)
        (Z9) edge                      node {$\epsilon,\circ\in\Gamma/\circ\circ$} (Z1)
        (Z10) edge [loop right]         node{$\epsilon,\star\in\Gamma/\epsilon$}   (Z10)
        ;
\end{tikzpicture}
\end{center}



\section*{Aufgabe 6.2}
\scriptsize
\textbf{Regeln $\forall Z_i \in Z: S \rightarrow R_{(Z_S,\#,Z_i)}$\\}
\begin{tabular}{l}
$S \rightarrow R_{(Z_1,\#,Z_1)}$ \\
$S \rightarrow R_{(Z_1,\#,Z_2)}$ \\
\end{tabular}

\textbf{Regeln für Kanten $a,b/\epsilon$\\}
\begin{tabular}{l}
$R_{(Z_1,b,Z_1)} \rightarrow b$ \\
$R_{(Z_1,\#,Z_1)} \rightarrow b$ \\
$R_{(Z_1,c,Z_1)} \rightarrow c$ \\
\end{tabular}

\textbf{Regeln für Kanten $a,b/c$\\}
\begin{tabular}{l}
$R_{(Z_1,b,Z_1)} \rightarrow bR_{(Z_1,b,Z_1)}$ \\
$R_{(Z_1,b,Z_2)} \rightarrow bR_{(Z_1,b,Z_2)}$ \\
$R_{(Z_2,b,Z_1)} \rightarrow cR_{(Z_2,b,Z_1)}$ \\
$R_{(Z_2,b,Z_2)} \rightarrow cR_{(Z_2,b,Z_2)}$ \\
$R_{(Z_2,b,Z_1)} \rightarrow bR_{(Z_1,c,Z_1)}$ \\
$R_{(Z_2,b,Z_2)} \rightarrow bR_{(Z_1,c,Z_2)}$ \\
\end{tabular}

\textbf{Regeln für Kanten $a,b/cd$\\}
\begin{tabular}{l}
$R_{(Z_1,\#,Z_1)} \rightarrow aR_{(Z_2,b,Z_1)}R_{(Z_1,\#,Z_1)}$ \\
$R_{(Z_1,\#,Z_1)} \rightarrow aR_{(Z_2,b,Z_2)}R_{(Z_2,\#,Z_1)}$ \\
$R_{(Z_1,\#,Z_2)} \rightarrow aR_{(Z_2,b,Z_1)}R_{(Z_1,\#,Z_2)}$ \\
$R_{(Z_1,\#,Z_2)} \rightarrow aR_{(Z_2,b,Z_2)}R_{(Z_2,\#,Z_2)}$ \\
\end{tabular}

\clearpage
\textbf{Regelset mit markierten möglichen Regeln}\\
\begin{tabular}{lll}
\colorbox{green!30}{$S \rightarrow R_{(Z_1,\#,Z_1)}$} & \colorbox{green!30}{$R_{(Z_1,b,Z_1)} \rightarrow bR_{(Z_1,b,Z_1)}$} & \colorbox{green!30}{$R_{(Z_1,\#,Z_1)} \rightarrow aR_{(Z_2,b,Z_1)}R_{(Z_1,\#,Z_1)}$}\\
\colorbox{white!30}{$S \rightarrow R_{(Z_1,\#,Z_2)}$} & \colorbox{white!30}{$R_{(Z_1,b,Z_2)} \rightarrow bR_{(Z_1,b,Z_2)}$} & \colorbox{white!30}{$R_{(Z_1,\#,Z_1)} \rightarrow aR_{(Z_2,b,Z_2)}R_{(Z_2,\#,Z_1)}$}\\
\colorbox{green!30}{$R_{(Z_1,b,Z_1)} \rightarrow b$}  & \colorbox{green!30}{$R_{(Z_2,b,Z_1)} \rightarrow cR_{(Z_2,b,Z_1)}$} & \colorbox{white!30}{$R_{(Z_1,\#,Z_2)} \rightarrow aR_{(Z_2,b,Z_1)}R_{(Z_1,\#,Z_2)}$}\\
\colorbox{green!30}{$R_{(Z_1,\#,Z_1)} \rightarrow b$} & \colorbox{white!30}{$R_{(Z_2,b,Z_2)} \rightarrow cR_{(Z_2,b,Z_2)}$} & \colorbox{white!30}{$R_{(Z_1,\#,Z_2)} \rightarrow aR_{(Z_2,b,Z_2)}R_{(Z_2,\#,Z_2)}$}\\
\colorbox{green!30}{$R_{(Z_1,c,Z_1)} \rightarrow c$}  & \colorbox{green!30}{$R_{(Z_2,b,Z_1)} \rightarrow bR_{(Z_1,c,Z_1)}$} & \\
& \colorbox{white!30}{$R_{(Z_2,b,Z_2)} \rightarrow bR_{(Z_1,c,Z_2)}$} & 
\end{tabular}

Die entstehende Grammatik ist:
\begin{align*}
S &\rightarrow R_{(Z_1,\#,Z_1)} \\
R_{(Z_1,b,Z_1)} &\rightarrow b \\
R_{(Z_1,\#,Z_1)} &\rightarrow b \\
R_{(Z_1,c,Z_1)} &\rightarrow c \\
R_{(Z_1,b,Z_1)} &\rightarrow bR_{(Z_1,b,Z_1)} \\
R_{(Z_2,b,Z_1)} &\rightarrow cR_{(Z_2,b,Z_1)} \\
R_{(Z_2,b,Z_1)} &\rightarrow bR_{(Z_1,c,Z_1)} \\
R_{(Z_1,\#,Z_1)} &\rightarrow aR_{(Z_2,b,Z_1)}R_{(Z_1,\#,Z_1)}
\end{align*}
Gekürzt mit $R_1 = R_{(Z_1,\#,Z_1)}, R_2 = R_{(Z_1,b,Z_1)}, R_3 = R_{(Z_1,c,Z_1)}, R_4 = R_{(Z_2,b,Z_1)}$:
\begin{align*}
S   &\rightarrow R_1               \\
R_1 &\rightarrow aR_4R_1 \ |\ b    \\
R_2 &\rightarrow bR_2    \ |\ b    \\
R_3 &\rightarrow c                 \\
R_4 &\rightarrow cR_4    \ |\ bR_3
\end{align*}
\textit{Anm.: $R_2$ ist nicht erreichbar.}
\normalsize



\section*{Aufgabe 6.3}
\begin{algorithm}[H]
  \KwIn{graph $G$, vertexlist $V$, empty path $P$, empty cycle $C$}
  \KwOut{contracted graph $G$}
  initialize $V$ with all vertices of $G$\;
  \While{$V \neq \emptyset$}{
    \Repeat{$\nexists v \in V$ with inDeg == 0 or outDeg == 0}{
      set $Q = \forall v \in V\ with\ inDeg\ ==\ 0\ or\ outDeg\ ==\ 0$\;
      \ForEach{$q \in Q$}{
        remove $q$ from $V$\;
      }
    }
    \tcc{If there are vertices left, there must be a cycle}
    \If{$V \neq \emptyset$}{
      pick $v$ from $V$\;
      \While{$v \notin P$}{
        add $v$ to $P$\;
        $v$ = follower of $v$\;
      }
      $C = P$ from $v$ to $end$\;
      new vertex $q = contractCycle(C)$\;
      remove $C$ from $G$\;
      add $q$ to $G$\;
      empty $P,C$\;
    }
  }
  \caption{Contract Graph}
\end{algorithm}

\begin{algorithm}[H]
  \KwIn{cycle $C$}
  \KwOut{vertex $q$}
  initialize vertex $q$\;
  \ForEach{$v \in C$}{
    \ForEach{incoming arc of $v$} {
      end arc at $q$;
    }
    \ForEach{outgoing arc of $v$} {
      start arc at $q$;
    }
  }
  \Return{q}
  \caption{Contract Cycle}
\end{algorithm}

\begin{itemize}
	\item \textit{Algorithmus 1: Graph kontrahieren}: $\Omega(1 \cdot n) = \Omega(n)$ und $\mathcal{O}(n \cdot (n \cdot 1)) = \mathcal{O}(n^2)$
  \item \textit{Algorithmus 2: Kreis kontrahieren}: $\mathcal{O}(n)$
\end{itemize}



\section*{Aufgabe 6.4}
Wir befolgen die einzelnen Schritte der Vorlesung, um die Grammatik in CNF umzuwandeln.
\subsection*{Schritt 1 - Symbole ausgliedern}
Hier ersetzt man einfach die Terminale mit Variablen.
\begin{align*}
S&\rightarrow D | D_dA\\
A&\rightarrow A_aS | S | E\\
B&\rightarrow D_dB_b | C\\
C&\rightarrow E_eED_d | B\\
D&\rightarrow A | CD_dDE_eB | D_d\\
E&\rightarrow B | BB | C\\
A_a&\rightarrow a\\
B_b&\rightarrow b\\
D_d&\rightarrow d\\
E_e&\rightarrow e
\end{align*}
\subsection*{Schritt 2 - Kreise und Senken entfernen}
\subsubsection*{a) Kreise loswerden}
Hier kann man den Graphen für die $V \rightarrow V$ malen. Dann kommt raus:
\begin{align*}
S&\rightarrow D\rightarrow A\rightarrow S\\
B&\rightarrow C\rightarrow B
\end{align*}
Also ersetzen wir D,A durch S und C durch B. Das macht:
\begin{align*}
S&\rightarrow S | D_dS | A_aS | E | BD_dSE_eB | D_d\\
B&\rightarrow D_dB_b | B | E_eED_d\\
E&\rightarrow B | BB\\
A_a&\rightarrow a\\
B_b&\rightarrow b\\
D_d&\rightarrow d\\
E_e&\rightarrow e
\end{align*}

\subsubsection*{b) Senken loswerden}
Erst mal können die Selbstreferenzen alle entfernt werden. Dann muss nur noch das einzelne E und $D_d$ in S und das B in E angepasst werden.
\begin{align*}
S&\rightarrow D_dS | A_aS | D_dB_b | E_eED_d | BB | BD_dSE_eB | d\\
B&\rightarrow D_dB_b | E_eED_d\\
E&\rightarrow D_dB_b | E_eED_d | BB\\
A_a&\rightarrow a\\
B_b&\rightarrow b\\
D_d&\rightarrow d\\
E_e&\rightarrow e
\end{align*}

\subsection*{Schritt 3 - Variablen erfinden bis es passt}
Hier muss man nur noch die Regeln die auf mehr als zwei Variablen abbilden anpassen, sodass diese auch nur noch auf zwei Abbilden. Dazu werden nach belieben neue Variablen eingefügt.
\begin{align*}
S&\rightarrow D_dS | A_aS | D_dB_b | E_eX | BB | BY | d\\
B&\rightarrow D_dB_b | E_eX\\
E&\rightarrow D_dB_b | E_eX | BB\\
X&\rightarrow ED_d \\
Y&\rightarrow D_dZ \\
Z&\rightarrow SW \\
W&\rightarrow E_eB \\
A_a&\rightarrow a\\
B_b&\rightarrow b\\
D_d&\rightarrow d\\
E_e&\rightarrow e
\end{align*}



\section*{Aufgabe 6.5}
Sei $L = \left\{\alpha 2 \alpha 2 \alpha | \alpha \in \left\{0,1\right\}^* \right\} \subset \left\{0,1,2\right\}^*$ kontextfrei. Dann existiert gibt es für jedes Wort $z, |z|\geq n$ mit einer Zerlegung $z=uvwxy$ mit $|vx|\geq 1, |uvw|\leq n, \forall i \geq 0: uv^iwx^iy \in L$.

Betrachten wir das Wort $z = \alpha 2 \alpha 2 \alpha$ mit $|\alpha|=n$.

Dann ist $uvw \subseteq \alpha$. 

Egal wie wir das Wort aufpumpen: entweder wir verändern nur einen Teil der $\alpha$s, wodurch diese nicht mehr gleich sind, oder wir pumpen gar etwas mit 2en auf.



\section*{Aufgabe 6.6}
Unser Algorithmus geht nach folgenden Schritten vor:
\subsection*{Schritt 1 - Epsilons entfernen}
Zunächst entfernen wir alle Regeln der Form $V\rightarrow \epsilon$. Dazu entfernen wir einfach diese Regeln und wenn V sogar nur auf $\epsilon$ abbildet können wir die ganze Variable entfernen. Eine Ausnahme bildet hier die Startvariable, da wir diese nicht entfernen dürfen.

\subsection*{Schritt 2 - Selfloops entfernen}
Wir entfernen als nächstes alle Selfloops.

\subsection*{Schritt 3 - Unpassendes ersetzen}
Nach diesem Schritt bleiben nur noch Regeln der Form:
\begin{enumerate}
  \item $V \rightarrow V$
  \item $V \rightarrow V\times V$
  \item $V \rightarrow \Sigma$
  \item $V \rightarrow \Sigma \times V$
  \item $V \rightarrow \Sigma \times V \times V$
\end{enumerate}
Wobei nur die ersten beiden nicht der GNF entsprechen. Um diese umzuformen ersetzen wir iterativ die erste Variable bei den kritischen Regeln durch die ausgeschriebene Form der Variablen. Dann überprüfen wir, ob es nach dem Ersetzungsschritt immer noch kritische Regeln gibt und ersetzen noch einmal. usw.

Dieses Vorgehen funktioniert, da wir am Anfang nicht nur Selfloops sondern insbesondere auch $\epsilon$-Terminale entfernt haben. Wir landen also bei den Ersetzungen in jedem Fall am Ende immer auf einem $t \in \Sigma$.

\newpage
\subsection*{Beispiel}
\begin{align*}
S &\rightarrow \epsilon \ |\ AB \ |\ cA\\
A &\rightarrow AC \ |\ cBB\\
B &\rightarrow BB\ |\ cAD\\
C &\rightarrow \epsilon\\
D &\rightarrow f \ |\ \epsilon
\end{align*}
Jetzt wenden wir den ersten Schritt an und entfernen die $\epsilon$-Kanten.
\begin{align*}
\begin{array}{l|l}
S \rightarrow \epsilon \ |\ AB \ |\ cA                                 & S \rightarrow \epsilon \ |\ AB \ |\ cA \\ 
A \rightarrow {\color{red}A}\text{\cancel{\color{red}$C$}} \ |\ cBB    & A \rightarrow A\ |\ cBB                \\
B \rightarrow BB\ |\ {\color{red}cAD\ \underline{|\ cA}}               & B \rightarrow BB\ |\ cAD\ |\ cA        \\
\text{\cancel{${\color{red}C \rightarrow \epsilon}$}}                  & D \rightarrow f                        \\
{\color{red}D \rightarrow f}\ \text{\cancel{\color{red}$|\ \epsilon$}} & 
\end{array}
\end{align*}
Und jetzt Schritt zwei, um die Selfloops loszuwerden:
\begin{align*}
\begin{array}{l|l}
S \rightarrow \epsilon \ |\ AB \ |\ cA               & S \rightarrow \epsilon \ |\ AB \ |\ cA \\
A \rightarrow \text{\cancel{\color{red}$A\ |\ $}}cBB & A \rightarrow cBB                      \\
B \rightarrow BB\ |\ cAD\ |\ cA                      & B \rightarrow BB\ |\ cAD\ |\ cA        \\
D \rightarrow f                                      & D \rightarrow f
\end{array}
\end{align*}
Und zuletzt die Ersetzungen:
\begin{align*}
\begin{array}{l|l}
S \rightarrow \epsilon \ |\ \frac{\text{\cancel{\color{red}$A$}}}{\color{red}cBB}B \ |\ cA & S \rightarrow \epsilon \ |\ cBBB \ |\ cA  \\
A \rightarrow cBB                                                                          & A \rightarrow cBB                         \\
B \rightarrow \frac{\text{\cancel{\color{red}$B$}}}{\color{red}cAD|cA}B\ |\ cAD\ |\ cA     & B \rightarrow cADB\ |\ cAB\ |\ cAD\ |\ cA \\
D \rightarrow f                                                                            & D \rightarrow f
\end{array}
\end{align*}
Damit sind alle Regeln nun in GNF.

\subsection*{Anmerkung}
Schritte 1 und 2 sind notwendig, da nur das Ersetzen nicht ausreichend ist. Man nehme z.B. die einfache Grammatik:
\begin{align*}
S &\rightarrow a\ |\ CS\ |\ SS \\
C &\rightarrow \epsilon
\end{align*}
Nur die Ersetzungen führen zu:
\begin{align*}
S &\rightarrow a\ |\ \epsilon S\ |\ aS\ |\ SS \\
C &\rightarrow \epsilon
\end{align*}
Das ist problematisch in den Regeln $S \rightarrow \epsilon S$ und $S \rightarrow SS$.

\end{document}
